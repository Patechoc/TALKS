%%%%%%%%%%%%%%%%%%%%%%%%%%%%%%%%%%%%%%%%%%%%%%%%%%%%%%%%%%%%%%%%%%%%%%%%%%%%%%%%%%%%%%%%%%%%%%%%%%%%%
%%%%% slide 1 
\begin{frame}{Outline}
\footnotesize

\begin{itemize}
\item The Resolution-of-the-Identity (RI) approximation
\item Methods exploiting locality to reach linear-scaling RI
\item The PARI approximation
\item Results and Analysis of convergence issues
\item Conclusion
\end{itemize}

\end{frame}
%%%%%%%%%%%%%%%%%%%%%%%%%%%%%%%%%%%%%%%%%%%%%%%%%%%%%%%%%%%%%%%%%%%%%%%%%%%%%%%%%%%%%%%%%%%%%%%%%%%%%
%%%%% slide 2
\frametitle{The RI approximation}
\framesubtitle{}

\begin{frame}{The RI approximation}
\footnotesize

 The Resolution-of-the-Identity (RI), or the {\blue density-fitting} approximation:
\begin{itemize}
  \item Speed up calculations ({\blue typically by factor 3-30}) at little loss of chemical accuracy
  \item Highly successful for {\blue approximating the Coulomb contribution} (in particular in DFT)
  \item Also used for small- to medium-sized systems for {\blue approximation of the exact exchange} and 
        in correlated treatment - {\red Scaling wall}
  \item {\red Requires an auxiliary basis set}
\end{itemize}

\end{frame}

%%%%%%%%%%%%%%%%%%%%%%%%%%%%%%%%%%%%%%%%%%%%%%%%%%%%%%%%%%%%%%%%%%%%%%%%%%%%%%%%%%%%%%%%%%%%%%%%%%%%%
%%%%% slide 3
\frametitle{The RI approximation}
\framesubtitle{}

\begin{frame}{The RI approximation}
\footnotesize
\begin{itemize}
\item In the most commonly used RI treatments, the
     four-center two-electron integrals (in the Mulliken notation)
\beq
  (\mu\nu|\gamma\delta) = \int_{{\mathbb R}^3} \int_{{\mathbb R}^3} \psi_\mu(\mathbf{r}_1)\psi_\nu(\mathbf{r}_1) \frac{1}{r_{12}} \psi_\gamma(\mathbf{r}_2)\psi_\delta(\mathbf{r}_2)
                    d\mathbf{r}_1d\mathbf{r}_2.
\eeq
 are 
     approximated by ${\blue \widetilde{(\mu\nu|\gamma\delta)}}$, according to
%
\beq
  (\mu\nu|\gamma\delta) \approx \widetilde{(\mu\nu|\gamma\delta)} 
   = \sum_{IJ} (\mu\nu|I)(I|J)^{-1}(J|\gamma\delta)
\eeq
%
with ${\blue \left\{ \mu \right\}}$ the (regular) AO basis set and with
an atom-centered auxiliary basis set ${\blue \left\{ I \right\}}$ 

  \item The Coulomb contribution ({\blue up to 10 000 AO basis functions})
\beq
\begin{split}
  &J_{\mu\nu} = (\mu\nu|\rho) = \sum_{\gamma\delta} (\mu\nu|\gamma\delta) D_{\gamma\delta}\\
  \approx &\tilde J_{\mu\nu} = (\mu\nu|\tilde\rho) = \sum_{I} (\mu\nu|I)c_I, 
   \quad \red \sum_J (I|J)c_J = (J|\rho)
\end{split}
\eeq


\end{itemize}
\end{frame}

%%%%%%%%%%%%%%%%%%%%%%%%%%%%%%%%%%%%%%%%%%%%%%%%%%%%%%%%%%%%%%%%%%%%%%%%%%%%%%%%%%%%%%%%%%%%%%%%%%%%%
%%%%% slide 4
\frametitle{}
\framesubtitle{}

\begin{frame}{RI for the exact exchange (RI-K)}
\footnotesize
\begin{itemize}
\item The (regular) exchange contribution
\begin{eec}
  K_{\mu\nu} = \sum_{\gamma\delta} (\mu\gamma|\nu\delta) D_{\gamma\delta} = \sum_i^\text{occ} (\mu i | \nu i)
\end{eec}
approximated according to - {\blue Weigend (2002), Polly~\emph{et.al} (2004)}
\begin{eec}
  \tilde K_{\mu\nu} = \sum_{\gamma\delta} \sum_{IJ} (\mu\gamma|I)(I|J)^{-1}(J|\nu\delta) D_{\gamma\delta}
                    = \sum_i^\text{occ} \sum_{IJ} (\mu i |I)(I|J)^{-1}(J| \nu i)
\end{eec}
\item {\red Scaling wall} limits this approach to about 1000 AO basis-functions

\begin{itemize}
  \footnotesize 
  \item metric matrix $\blue (I|J)$ non-sparse $\red \to {\cal O}(N^3)$ scaling
  \item non-locality of fitting coefficients $\blue c_I^{\mu\nu}$ (\red for each pair $\mu\nu$)
\begin{eec}
  \blue c_I^{\mu\nu} = \sum_J (I|J)^{-1} (J|\mu\nu)
\end{eec}
  \item transformation steps, e.g. 
\begin{eec}
  \blue (I|\mu i) = \sum_{\nu} (I|\mu\nu) L_{\nu i}, \quad {\red {\cal O}(N^2 N_\text{occ})}
\end{eec}
\end{itemize}
\end{itemize}
\end{frame}

%%%%%%%%%%%%%%%%%%%%%%%%%%%%%%%%%%%%%%%%%%%%%%%%%%%%%%%%%%%%%%%%%%%%%%%%%%%%%%%%%%%%%%%%%%%%%%%%%%%%%
%%%%% slide 5.1
\frametitle{}
\framesubtitle{}

\begin{frame}{Methods for linear-scaling RI}
\footnotesize

Two main approaches for linear-scaling RI:
\begin{itemize}
  \item {\red the Partitioning approach}
    \\Partition the density and fit each density in a local region \\{\blue Gallant and St-Amant (1996),
        Sodt~\emph{et.al} (2006), Sodt and Head-Gordon (2008)}
%\begin{eec}
%  \rho(\mathbf{r}) = \sum_s \rho^s(\mathbf{r}), \quad \red \rho^s(\mathbf{r}) \approx \tilde\rho^s(\mathbf{r})
%\end{eec}
  \item {\red the Local Metric approach}
    \\Obtain the fitting coefficients using (\textit{w}) a local metric
    \\(e.g. the overlap metric)
    \\{\blue Baerends~\emph{et.al} (1973), Vahtras~\emph{et.al} (1993), Jung~\emph{et.al} (2005), Reine~\emph{et.al} (2008)}
\begin{eec}
  c_I^{\mu\nu} = \sum_J \langle I|w|J \rangle^{-1} \langle J|w|\mu\nu\rangle
\end{eec}
instead of the Coulomb metric
\begin{eec}
  \blue c_I^{\mu\nu} = \sum_J (I|J)^{-1} (J|\mu\nu)
\end{eec}
\end{itemize}
\end{frame}

%%%%%%%%%%%%%%%%%%%%%%%%%%%%%%%%%%%%%%%%%%%%%%%%%%%%%%%%%%%%%%%%%%%%%%%%%%%%%%%%%%%%%%%%%%%%%%%%%%%%%
%%%%% slide 5.2
\frametitle{}
\framesubtitle{}

\begin{frame}{Methods for linear-scaling RI}
\footnotesize
%
{\blue Atomic RI (ARI)}: an elegant and balanced partitioning approach:
\begin{itemize}
\item In the ARI approach of Sodt~\emph{et.al} each charge distribution $\blue |\mu\nu)$ is 
     first associated with a center $\blue C$, and is fitted using auxiliary functions 
     $\blue |I)$ centered on the atoms in some buffer zone $\blue [C]$ around $\blue C$ 

\begin{eec}
  |\mu\nu) = \sum_I c_I^{\mu\nu} |I), \quad \red |I) \in [C]
\end{eec}
\begin{itemize}
  \footnotesize 
  \item fixed buffer-zone size - {\red system dependent}
  \item the set $\blue |I)$ includes the {\red auxiliary functions centered on several atoms}
\end{itemize}
\item Jung~\emph{et.al} explored the decay behavior of the two-center fitting coefficients
     $\blue c_I^{\mu\nu}$. 
\begin{itemize}
  \footnotesize 
    \item Coulomb metric fit - the asymptotic decay of $\blue c_I^{\mu\nu}$ was as slow 
          as $\red \frac{1}{r}$
    \item local ({\blue overlap/attenuated Coulomb}) metric fit - {\red exponential decay} with distance
    \item But the fitting errors $\approx 10$ times smaller in the Coulomb than in the overlap metric
\end{itemize}
\end{itemize}

\end{frame}


%%%%%%%%%%%%%%%%%%%%%%%%%%%%%%%%%%%%%%%%%%%%%%%%%%%%%%%%%%%%%%%%%%%%%%%%%%%%%%%%%%%%%%%%%%%%%%%%%%%%%
%%%%% slide 6.1
\frametitle{}
\framesubtitle{}

\begin{frame}{The PARI approach}
\footnotesize

We propose the pair-atomic resolution-of-the-identity (PARI):
\begin{itemize}
\item fit each charge distribution $\blue |\mu\nu)$ with auxiliary functions
     $\blue |I)$ centered on the two parent atoms $\blue A$ and $\blue B$ of 
     the basis functions $\blue |\mu)$ and $\blue |\nu)$
\begin{eec}
  |\widetilde{\mu\nu}) = \sum_{I\in{A\cup B}} c_I^{\mu\nu}|I), 
   \quad c_I^{\mu\nu} = \sum_{\red J\in{A\cup B}} (I|J)^{-1}(J|\mu\nu)
\end{eec}
instead of the (regular) RI approx.
\begin{eec}
  |\widetilde{\mu\nu}) = \sum_{I\in{\cal M}} c_I^{\mu\nu}|I), 
   \quad c_I^{\mu\nu} = \sum_{\red J\in{\cal M}} (I|J)^{-1}(J|\mu\nu)
\end{eec}
\item replace all $\blue (\mu\nu|\gamma\delta)$ by {\red robust} and {\red variational} 
      $\blue \widetilde{(\mu\nu|\gamma\delta)}$ given by
\begin{equation*}
\begin{split}
\blue
  \widetilde{(\mu\nu|\gamma\delta)} &\blue= (\mu\nu|\widetilde{\gamma\delta}) 
    + (\widetilde{\mu\nu}|\gamma\delta) - (\widetilde{\mu\nu}|\widetilde{\gamma\delta})\\
& = \blue (\mu\nu|\gamma\delta) - \red (\mu\nu-\widetilde{\mu\nu}|\gamma\delta-\widetilde{\gamma\delta})
\end{split}
\end{equation*}
\item Pros/Cons:
\begin{itemize}
\item {\blue highly local}
\item {\blue removes scaling bottleneck }
\item {\red decreases accuracy} $\to$ larger auxiliary basis sets
\end{itemize}
\end{itemize}

\end{frame}
%
%%%%%%%%%%%%%%%%%%%%%%%%%%%%%%%%%%%%%%%%%%%%%%%%%%%%%%%%%%%%%%%%%%%%%%%%%%%%%%%%%%%%%%%%%%%%%%%%%%%%%
%%%%% slide 6.2
\frametitle{}
\framesubtitle{}

\begin{frame}{PARI-J}
\footnotesize

PARI for the Coulomb contribution (PARI-J)
\begin{equation*}
\begin{split}
  \widetilde{J}_{\mu\nu} &=  \sum_{\gamma\delta} \widetilde{(\mu\nu|\gamma\delta)} D_{\gamma\delta} 
  = \sum_{\gamma\delta} \left[  (\widetilde{\mu\nu}|\gamma\delta) +  (\mu\nu|\widetilde{\gamma\delta}) -  (\widetilde{\mu\nu}|\widetilde{\gamma\delta}) \right]  D_{\gamma\delta}\\
   &=  \sum_{\gamma\delta} \left[ \sum_{I\in{A\cup B}}c_I^{\mu\nu} (I|\gamma\delta) + \sum_{J\in{C\cup D}} (\mu\nu|J) c_J^{\gamma\delta} - \sum_{I\in{A\cup B},J\in{C\cup D}}c_I^{\mu\nu} (I|J) c_J^{\gamma\delta} \right]  D_{\gamma\delta}\\
   &=  \sum_{I\in{A\cup B}}c_I^{\mu\nu} (I|\rho) + \sum_{J\in{C\cup D}} (\mu\nu|J) c_J - \sum_{I\in{A\cup B}}c_I^{\mu\nu} (I| \tilde{\rho}),  with \quad \red c_J = \sum_{\gamma\delta} c_J^{\gamma\delta} D_{\gamma\delta} \\
 &= \sum_{J\in{C\cup D}} (\mu\nu|J) c_J +  \sum_{I\in{A\cup B}}c_I^{\mu\nu} (I|\rho-\tilde{\rho}), with \quad 
\blue |\tilde\rho) = \sum_J c_J|J),  \red \sum_J (I|J)c_J = (J|\rho) \\
   & = \sum_I (\mu\nu|I)c_I + \sum_{I\in{A\cup B}} C_I^{\mu\nu}(I|\rho-\tilde\rho)
\end{split}
\end{equation*}
\begin{itemize}
  \item straightforwardly linear scaling using screening and FMM
\end{itemize}
\end{frame}
%

%%%%%%%%%%%%%%%%%%%%%%%%%%%%%%%%%%%%%%%%%%%%%%%%%%%%%%%%%%%%%%%%%%%%%%%%%%%%%%%%%%%%%%%%%%%%%%%%%%%%
%%%%% slide 6.3
\frametitle{}
\framesubtitle{}

\begin{frame}{PARI-K}
\footnotesize

PARI for the Exact-Exchange contribution (PARI-K)
\begin{itemize}
\item split fitting coefficients into the two parent atoms
\begin{eec}
  |\widetilde{\mu\nu}) = 
  |\widetilde{\underbar{$\mu$}\nu}) + |\widetilde{\mu\underbar{$\nu$}}), 
  \quad \red |\widetilde{\underbar{$\mu$}\nu}) = \sum_{I\in A} c_I^{\mu\nu} |I),
  \quad \blue |\mu\widetilde{\underbar{$\nu$}}) = \sum_{J\in B} c_J^{\mu\nu} |J)
\end{eec}
\item leads to a splitting of $\blue \widetilde{(\mu\nu|\gamma\delta)}$ into eight parts
      - four involving three-center and four two-center integrals
\item collect terms in order to calculate each integral only once and select 
      contraction pathway in order to reduce cost
\item exploit symmetry of integrals and density matrix 
\end{itemize}
An example three-center term:
\begin{equation*}
 \blue K_{\mu\gamma} \stackrel{+}{=} \sum\limits_{I\delta} d_{I,\delta}^\mu (I|\gamma\delta), 
  \quad \red I,\mu\in A,\blue \gamma\in C, \red \delta\in {\cal M}, \quad
  \blue d_{I,\gamma}^\mu = \sum\limits_\nu c_I^{\mu\nu}D_{\nu\gamma}
%\begin{split}
% \blue K_{\mu\gamma} &\blue\stackrel{+}{=} \sum\limits_{I\delta} d_{I,\delta}^\mu (I|\gamma\delta), \quad \red I,\mu\in A,\blue \gamma\in C, \red \delta\in {\cal M}\\
% \blue d_{I,\gamma}^\mu &\blue = \sum\limits_\nu c_I^{\mu\nu}D_{\nu\gamma}
%\end{split}
\end{equation*}
\begin{itemize}
  \item ignoring screening each of these terms scale $\blue {\cal O}(N^3 n^3 a)$\\
    number of atoms $\blue N$, of regular/auxiliary basis functions per atom 
        $\blue n$,$\blue a$
  \item to be compared with the $\red {\cal O}(N^4 n^2 a^2)$ scaling of conventional RI
%- {\blue can be reduced to $\red {\cal O}(N^3 N_\text{occ} n a^2)$ using MOs, with $\red N_\text{occ}$ the number of occupied MOs}
\end{itemize}
\end{frame}
%
%%%%%%%%%%%%%%%%%%%%%%%%%%%%%%%%%%%%%%%%%%%%%%%%%%%%%%%%%%%%%%%%%%%%%%%%%%%%%%%%%%%%%%%%%%%%%%%%%%%%%
%%%%% slide 7
\frametitle{}
\framesubtitle{}

\begin{frame}{ Outline of PARI-K algorithm}
\footnotesize
\begin{figure}
\begin{tabbing}
\hspace{3pt}\=\hspace{10pt}\=\hspace{10pt}\=\hspace{10pt}\=\hspace{10pt}\=\hspace{10pt}\=\hspace{10pt}\=\hspace{10pt}\=\hspace{10pt}\=\hspace{10pt}\kill
\> {\blue Three-center contributions} \\
\> Loop $A$ \\
\> \> Calculate the fitting coefficients $\blue c_I^{\mu\nu}, \quad \red I,\mu\in A, \blue \nu\in {\cal M}$ \\
\> \> and density coefficients $\blue d_{I,\gamma}^\mu $ \\
%\> \> Construct density coefficients $\blue d_{I,\gamma}^\mu $\\
\> \> Loop $C$ \\
\> \> \> Loop $D\ge C$ \\
\> \> \> \> Calculate $\blue (I|\gamma\delta), \quad \red I\in A,\blue\gamma\in C, \red\delta\in D$ \\
\> \> \> \> Make contractions \\
\> \> \> \> \> $\blue \tilde K_{\mu\gamma} \stackrel{+}{=} \sum\limits_{I\delta} d_{I,\delta}^\mu (I|\gamma\delta), \red \quad {\cal O}(N^3 n^3 a)$ \\
\> \> \> \> \> $\blue \tilde K_{\mu\delta} \stackrel{+}{=} \sum\limits_{I\gamma} d_{I,\gamma}^\mu (I|\gamma\delta), \red \quad {\cal O}(N^3 n^3 a)$ \\
\> \> \> \> \> $\blue X_{I,\gamma}^\mu = \sum\limits_\delta (I|\gamma\delta)D_{\mu\delta}, \red \quad {\cal O}(N^3 n^3 a)$ \\
\> \> \> \> \> $\blue X_{I,\delta}^\mu = \sum\limits_\gamma (I|\gamma\delta)D_{\mu\gamma}, \red \quad {\cal O}(N^3 n^3 a)$ \\
%\> \> \> End loop $D$ \\
\> \> Make contractions \\
\> \> \> $\blue \tilde K_{\nu \gamma} \stackrel{+}{=} \sum_{I,\mu} c_{I}^{\mu\nu} X_{I,\gamma}^\mu, \red \quad {\cal O}(N^3 n^3 a)$ \\
\> \> \> $\blue \tilde K_{\nu \delta} \stackrel{+}{=} \sum_{I,\mu} c_{I}^{\mu\nu} X_{I,\delta}^\mu, \red \quad {\cal O}(N^3 n^3 a)$ \\
%\> \> End loop $C$ \\
%\> End loop $A$ \\
\> Symmetrized or anti-symmetrized $\blue \mathbf{\tilde K}$ \\
\end{tabbing}
\end{figure}
\end{frame}


%%%%%%%%%%%%%%%%%%%%%%%%%%%%%%%%%%%%%%%%%%%%%%%%%%%%%%%%%%%%%%%%%%%%%%%%%%%%%%%%%%%%%%%%%%%%%%
%%%%% slide 8
\begin{frame}{Results and Analysis of convergence issues}
\footnotesize



\begin{center}
G2/G3   Benchmark set (358 molecules), B3LYP, 6-31G/df-def2
\begin{tabular}{lcccccc}
\hline 
           & $\delta (dfJ)$ & $\delta (pariJ)$ & $\delta (dfK)$ & $\delta (pariK)$ & $\delta (dfJK)$ & $\delta (pariJK)$ \\ 
           $\mu$  & -2.14e-05  & -7.40e-05  &  8.04e-06 & -3.21e-05 & -1.33e-05  & -5.34e-05  \\
          $\sigma$   &  9.11e-06  &  4.82e-05  & -1.11e-06 &  3.48e-05 &  8.00e-06  &  4.39e-05  \\
          $\sigma^{max}$  &  5.14e-05  &  2.05e-04  &  1.45e-05 &  1.35e-04 &  3.68e-05  &  1.86e-04 \\

\hline
\end{tabular}
\end{center}

Some convergence issues:
\begin{itemize}
\item some molecules have a diverging negative energy during SCF
\item we observed the non-positive definitness of the PARI approx.
\item this may come from the non-uniformity of the PARI operator, 
\item leading to discontinuities of the Potential Energy Surface
\end{itemize}

\end{frame}


%%%%%%%%%%%%%%%%%%%%%%%%%%%%%%%%%%%%%%%%%%%%%%%%%%%%%%%%%%%%%%%%%%%%%%%%%%%%%%%%%%%%%%%%%%%%%%
%%%%% slide ...
\begin{frame}{Results and Analysis of convergence issues}
\footnotesize

\begin{center}
Timings for 6-31G\_df-def2 (in sec. per iteration)
\begin{tabular}{lcccccc}
\hline
& $regJ$ & $pariJ$ & $dfJ$ & $LinK$ & $pariK$ & $dfK$ \\ 
PA\_oligomer\_2 & 1.89 & 1.18 & 0.68 & 2.88 & 9.5 & 11.11 \\ 
PA\_oligomer\_3 & 0.24 & 0.28 & 0.12 & 0.37 & 1.18 & 0.63 \\ 
PA\_oligomer\_4 & 0.63 & 0.51 & 0.27 & 0.97 & 3.21 & 2.02 \\ 
PA\_oligomer\_5 & 1.18 & 0.81 & 0.43 & 1.98 & 6.15 & 5.07 \\ 
PP & 2.71 & 1.48 & 0.84 & 4.83 & 11.34 & 12.33 \\ 
acene\_1 & 1.93 & 1.08 & 0.61 & 3.35 & 6.9 & 5.76 \\ 
acene\_2 & 4.33 & 2.18 & 1.29 & 7.89 & 18.85 & 19.23 \\ 
acene\_3 & 8.69 & 3.46 & 1.92 & 15.91 & 34.93 & - \\ 
acene\_4 & 11.83 & 4.27 & 2.89 & 22.11 & 56.25 & - \\ 
acene\_5 & 19.82 & 5.69 & 3.96 & 38.14 & 76.0 & - \\ 
beta-dipeptide & 2.19 & 1.42 & 0.75 & 4.51 & 10.8 & 11.57 \\ 
co & 0.01 & 0.08 & 0.01 & 0.01 & 0.09 & 0.03 \\ 
dipeptide & 1.58 & 1.17 & 0.59 & 3.38 & 8.66 & 7.77 \\ 
dmabn & 2.66 & 1.54 & 0.88 & 5.63 & 12.03 & 13.26 \\ 
h2co & 0.02 & 0.12 & 0.02 & 0.03 & 0.22 & 0.07 \\ 
hcl & 0.01 & 0.13 & 0.01 & 0.01 & 0.14 & 0.03 \\ 
n2 & 0.01 & 0.06 & 0.01 & 0.01 & 0.08 & 0.03 \\ 
tripeptide & 3.42 & 1.98 & 1.14 & 7.17 & 20.64 & 18.24 \\ 
\hline
\end{tabular}
\end{center}

\end{frame}


%%%%% slide ...
\begin{frame}{Results and Analysis of convergence issues}
\footnotesize

\begin{center}
Timings for cc-pVTZ\_cc-pVTZdenfit (in sec. per iteration)
\begin{tabular}{lcccccc}
\hline
& $regJ$ & $pariJ$ & $dfJ$ & $LinK$ & $pariK$ & $dfK$ \\ 
PA\_oligomer\_2 & 138.00 & 10.80 & 5.18 & 395.00 & 148.00 & 202.87 \\ 
PA\_oligomer\_3 & 13.53 & 2.09 & 0.80 & 36.60 & 15.54 & 11.48 \\ 
PA\_oligomer\_4 & 41.06 & 4.33 & 1.86 & 116.00 & 38.99 & 40.04 \\ 
PA\_oligomer\_5 & 81.00 & 7.44 & 3.45 & 291.00 & 89.00 & 143.84 \\ 
PP & 210.00 & 13.33 & 6.62 & 675.00 & 157.00 & 295.73 \\ 
acene\_1 & 173.00 & 10.87 & 5.53 & 477.00 & 131.00 & 204.18 \\ 
acene\_2 & 423.00 & 20.55 & 11.22 & 1199.00 & 287.00 & - \\ 
acene\_3 & 799.00 & 33.21 & 18.35 & 2363.00 & 554.00 & - \\ 
acene\_4 & 1295.00 & 48.32 & 28.88 & 3746.00 & 1064.00 & - \\ 
acene\_5 & 1797.00 & 64.00 & 39.59 & 5555.00 & 1659.00 & - \\ 
beta-dipeptide & 162.00 & 12.29 & 6.19 & 573.00 & 168.00 & 210.43 \\ 
co & 0.22 & 0.28 & 0.05 & 0.53 & 0.55 & 0.29 \\ 
dipeptide & 106.00 & 9.91 & 4.66 & 378.00 & 123.00 & 132.06 \\ 
dmabn & 221.00 & 15.26 & 7.04 & 854.00 & 183.00 & 231.08 \\ 
h2co & 0.78 & 0.59 & 0.12 & 2.26 & 1.87 & 0.93 \\ 
hcl & 0.18 & 0.43 & 0.04 & 0.40 & 0.63 & 0.17 \\ 
n2 & 0.22 & 0.24 & 0.05 & 0.53 & 0.53 & 0.28 \\ 
tripeptide & 279.00 & 18.47 & 9.12 & 952.00 & 291.00 & - \\ 
\hline
\end{tabular}
\end{center}

\end{frame}





%%%%%%%%%%%%%%%%%%%%%%%%%%%%%%%%%%%%%%%%%%%%%%%%%%%%%%%%%%%%%%%%%%%%%%%%%%%%%%%%%%%%%%%%%%%%%%
%%%%% Slide 9

\begin{frame}{Conclusion and future perspective}
\footnotesize
\begin{itemize}
\item We have presented the PARI scheme for efficient calculation of the Coulomb 
     and exchange contributions to DFT
\item The PARI method can in general be used for any method involving four-center two-electron
     integrals
\item Linear scaling in nature, relatively small errors
\item Allows differentiated treatment by adding Lagrangian multipliers
\item The method has some inherent problems, but highly promising
\end{itemize}
\begin{center}
{\red Thank you for your attention! }
\end{center}
\end{frame}
%
%%%%%%%%%%%%%%%%%%%%%%%%%%%%%%%%%%%%%%%%%%%%%%%%%%%%%%%%%%%%%%%%%%%%%%%%%%%%%%%%%%%%%%%%%%%%%%%%%%%%%


